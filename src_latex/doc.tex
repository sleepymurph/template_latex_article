\documentclass[
    %% We are writing US English in Europe. So use the US English language
    %% setting, but use A4 paper.
    %% There may be other Europe-friendly customizations throughout the
    %% template, such as setting biblatex to print dates in long form
    %% (e.g. "Apr. 8 2017") to avoid confusion over MDY vs DMY.
    usenglish,
    a4paper,
    %% See the longer comment "On draft and final" in the macros-general.tex for
    %% details on using the draft and final options, and the conventions these
    %% macros follow.
    %final,
]{article}

\input{packages.tex}
\input{macros-general.tex}
\input{macros-doc.tex}
% Style:
%   Don't capitalize initials in acronym.
%       Exception: proper name (like of an organization, e.g. NASA, IETF).
%       Exception: if the acronym is not obvious.
%   Do capitalize the first letter of all descriptions.

% Fields review:
%   key                 identifier used to look up the entry
%   name                how term appears in glossary, defaults to key
%   text                how term appears in text, defaults to name
%   first               how term appears in text on first use, defaults to text
%   plural              specialized plural, defaults to text+'s'
%   firstplural         specialized plural on first use, defaults to first+'s'
%   symbol              associated math symbol, defaults to blank (\relax)
%   symbolplural
%   sort                hidden name used for sorting, defaults to name
%
%   see                 list of other glossary terms to refer to
%   parent              specify a parent entry

% Usage review:
%   Note that most commands have capitalization variants:
%       \Gls___ capitalize first letter
%       \GLS___ capitalize all
%       \Acr___ capitalize first letter
%       \ACR___ capitalize all
%
% Usages that check+set the use flag:
%   \gls{label}             % normal reference
%   \glspl{label}           % plural
%   \glsdisp{labe}{text}    % explicitly set text
%
% Usages that do not check or set the use flag:
%   \glslink{label}{text}   % explicitly set text
%   \glstext{label}         % use text/long form
%   \glsfirst{label}        % use first/short form
%   \glsplural{label}       % use long form plural
%   \glsfirstplural{label}  % use first form plural
%   \glssymbol{label}       % use the symbol key (useful for math)
%   \glsdesc{label}         % use description key
%
% Acronym uses that do not check or set the use flag:
%   \acrshort{label}        % use short form
%   \acrlong{label}         % use the long form with no abbreviation
%   \acrfull{label}         % use the full "long (short)" form
%   Each of these also have \acr___pl variants for plurals




\ifoptionfinal{%
    % Nothing
}{

\newglossaryentry{cyber-physical system}{
    name={cyber-physical system},
    description={A computational system that is tightly conjoined with the physical world},
}

\newacronym[]{ML}{ML}{machine learning}


% Example for creating another category of glossaries, for journals
% These declarations come first, then \makeglossaries, then glossary entries.
% This could easily be a separate file if it gets too large.
\altnewglossary{cons}{cons}{Journals and Conferences}
\makeglossaries

\newacronym[
    type={cons},
]{ACM}{ACM}{Association for Computing Machinery}
\glsunset{ACM} % Well-known enough in context of this paper

}

\makeglossaries


\title{
    DOCUMENT TITLE
}
\author{ME}
\author{YOU}
\affil{OUR ORG}
\author{THE OTHER GUY}
\affil{THEIR ORG}

\date{\today}


\begin{document}

\maketitle

% Include a plain-text abstract if it exists.
%
% Keeping the abstract in plain text makes it easier to copy-and-paste it into
% web forms when submitting to publishers or archives.
%
% To not use an abstract, delete abstract.txt, and remove it from the list of
% dependencies in the Makefile
%
\IfFileExists{abstract.txt}{%
    \begin{abstract}%
        % Also use paragraph skips in abstract area
        \noindent
        \setlength{\parskip}{.5\baselineskip}%
        \setlength{\parindent}{0pt}%
        \inputplaintext{abstract.txt}
    \end{abstract}
}{%
    % Nothing
}

\section{Document Content}

\towrite{Document content here}

\input{example_content.tex}
\section{Example Graphviz Diagrams}

For an example Graphviz diagram, see
\cref{example_graphviz_diagram}%
.
The digram was generated from the source given in
\cref{diagram_common,example_graphviz_diagram_source}%
.

% Captions should go below graphics for images and diagrams.
% HOWEVER they can go above graphs and listings, as a title.
% For proper linking, always put the label AFTER the caption.
\begin{figure}[h]
    \centering
    \includegraphics[width=0.5\textwidth]{../generated_components/example_graphviz_diagram}
    \caption{Example Graphviz (dot) diagram}
    \label{example_graphviz_diagram}
\end{figure}

\lstinputlisting[
    caption={M4 Macros included in Graphviz code},
    label=diagram_common,
]{../src_graphviz/diagram_common.m4.dot}
\lstinputlisting[
    caption={Example Graphviz code},
    label=example_graphviz_diagram_source,
]{../src_graphviz/example_graphviz_diagram.dot}

\section{Example Matplotlib Diagram}

For an example Matplotlib diagram, see
\cref{example_matplotlib_plot}%
.

% Captions should go below graphics for images and diagrams.
% HOWEVER they can go above graphs and listings, as a title.
% For proper linking, always put the label AFTER the caption.
\begin{figure}[h]
    \centering
    \includegraphics[width=0.8\textwidth]{../generated_components/example_data_multimeter}
    \caption{Example Matplotlib diagram from CSV data}
    \label{example_matplotlib_plot}
\end{figure}


\printglossaries

% Include all sources even if not cited
%\nocite{*}
\printbibliography[]

\listoftodos
%\listoffigures
%\lstlistoflistings

\documenthistory

\end{document}
