% Style:
%   Don't capitalize initials in acronym.
%       Exception: proper name (like of an organization, e.g. NASA, IETF).
%       Exception: if the acronym is not obvious.
%   Do capitalize the first letter of all descriptions.

% Fields review:
%   key                 identifier used to look up the entry
%   name                how term appears in glossary, defaults to key
%   text                how term appears in text, defaults to name
%   first               how term appears in text on first use, defaults to text
%   plural              specialized plural, defaults to text+'s'
%   firstplural         specialized plural on first use, defaults to first+'s'
%   symbol              associated math symbol, defaults to blank (\relax)
%   symbolplural
%   sort                hidden name used for sorting, defaults to name
%
%   see                 list of other glossary terms to refer to
%   parent              specify a parent entry

% Usage review:
%   Note that most commands have capitalization variants:
%       \Gls___ capitalize first letter
%       \GLS___ capitalize all
%       \Acr___ capitalize first letter
%       \ACR___ capitalize all
%
% Usages that check+set the use flag:
%   \gls{label}             % normal reference
%   \glspl{label}           % plural
%   \glsdisp{labe}{text}    % explicitly set text
%
% Usages that do not check or set the use flag:
%   \glslink{label}{text}   % explicitly set text
%   \glstext{label}         % use text/long form
%   \glsfirst{label}        % use first/short form
%   \glsplural{label}       % use long form plural
%   \glsfirstplural{label}  % use first form plural
%   \glssymbol{label}       % use the symbol key (useful for math)
%   \glsdesc{label}         % use description key
%
% Acronym uses that do not check or set the use flag:
%   \acrshort{label}        % use short form
%   \acrlong{label}         % use the long form with no abbreviation
%   \acrfull{label}         % use the full "long (short)" form
%   Each of these also have \acr___pl variants for plurals




\ifoptionfinal{%
    % Nothing
}{

\newglossaryentry{cyber-physical system}{
    name={cyber-physical system},
    description={A computational system that is tightly conjoined with the physical world},
}

\newacronym[]{ML}{ML}{machine learning}


% Example for creating another category of glossaries, for journals
% These declarations come first, then \makeglossaries, then glossary entries.
% This could easily be a separate file if it gets too large.
\altnewglossary{cons}{cons}{Journals and Conferences}
\makeglossaries

\newacronym[
    type={cons},
]{ACM}{ACM}{Association for Computing Machinery}
\glsunset{ACM} % Well-known enough in context of this paper

}
