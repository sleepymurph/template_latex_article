% Some general macros I tend to reuse in most documents
%
% One day I'll figure out how to write proper TeX packages.
%


%======================================================================
% On 'draft' and 'final'
%
% There are actually four states of draft and final:
%
%   - no marker:    ambiguous
%   - draft:        definitely draft
%   - final:        definitely final
%   - both markers: usually decided by which marker comes last (see doc for 'ifdraft')
%
% The question is what to do when there is no marker.
%
% This Stack Overflow answer is a good list of draft/final behavior in various
% packages: https://tex.stackexchange.com/a/49369
%
% I prefer to write with hyperref on and todonotes both turned on.
%
%   - Hyperref turns off with the draft marker, but stays on with no marker.
%   - Todonotes with the 'obeyFinal' option does the same.
%
% So general strategy is to treat no marker as the default writing mode, with
% draft output. These macros reflect that. They will hide draft-like content
% only if the 'final' marker is used.
%


%======================================================================
% List environments with tighter spacing
%

\newenvironment{tight_itemize}{
\begin{itemize}
  \setlength{\itemsep}{0pt}
  \setlength{\parskip}{0pt}
}{\end{itemize}}

\newenvironment{tight_enumerate}{
\begin{enumerate}
  \setlength{\itemsep}{0pt}
  \setlength{\parskip}{0pt}
}{\end{enumerate}}

\newenvironment{inline_enumerate}
{\begin{enumerate*}[label={(\roman*)}]}
{\end{enumerate*}}


%======================================================================
% Email links
%

\newcommand{\emailhref}[1]{\href{mailto:#1}{#1}}


%======================================================================
% Customized todo notes
%
% Built on: todonotes
%

% To-Write, points that should be written into the text
\newcommand\towrite[1]{
    \todo[inline,color=white,bordercolor=white]{\textcolor{blue}{Write: #1}}
}

% Feedback, a note with feedback from a reviewer
\newcommandx{\feedback}[3][1={Feedback},2={},usedefault]{%
    \todo[color=blue!40,#2]{#1: #3}}

% Ask, make a note to ask a reviewer about a specific issue
\newcommandx{\ask}[3][1=\relax,2={},usedefault]{%
    \todo[color=violet!40,#2]{Ask #1: #3}}



%======================================================================
% TODO-like notes that are not actually tasks
%
% This is for todo-like annotations that do not represent tasks to be done, but
% that are still useful to leave as in-document comments while in non-final
% mode.
%
% Examples: finished todos, general guidance, notes to reviewers
%
% These should stick to the margin, where they will have less impact on length
% and flow of the document.
%

\newcommandx*{\pseudotodo}[2][1={.},usedefault]{
    \ifoptionfinal{%
        % nothing
    }{{%
        \marginpar{
            \leavevmode\color{#1}%
            \rule{\linewidth}{0.4pt}
            \small{#2}
        }
    }}%
}

% draft note, a note to self or reviewers when checking the draft
\newcommand*{\draftnote}[1]{%
    \pseudotodo{#1}
}

% guidance, advice to follow when writing section or paragraph
\newcommand*{\guidance}[1]{%
    \pseudotodo[black!050]{\textsc{#1}}
}

% outline, to mark main points and flow of document
\newcommand*{\outline}[1]{%
    \pseudotodo[blue!080]{#1}
}



% Place the following line in document to disable all pseudo-todos
%\renewcommandx*{\pseudotodo}[2][1={.},usedefault]{}


%======================================================================
% Shortcuts for specific units of measurement
%
% Built on: siunitx (with 'binary-units' option)
%
\newcommand{\gib}{\gibi\byte}
\newcommand{\mib}{\mebi\byte}
\newcommand{\kib}{\kibi\byte}



%======================================================================
% Including a plain text as content, but not verbatim
%
% This is useful for including content that might need to be reused elsewhere
% outside of TeX, like an abstract that will need to be copied into forms.
%
% Unlike verbatim, this does not change to typewriter font nor keep white space
% verbatim. It still treats the text as prose and paragraphs. But it treats TeX
% control characters like %, &, $, etc. as normal text so you do not have to
% escape them.
%

\newenvironment{plaintext}
{%
    \begingroup%
        \catcode`\$=12%
        \catcode`\&=12%
        \catcode`\#=12%
        \catcode`\^=12%
        \catcode`\_=12%
        \catcode`\~=12%
        \catcode`\%=12}
{\endgroup}

\newcommand{\inputplaintext}[1]
{\begin{plaintext}\input{#1}\end{plaintext}}


%======================================================================
% Scratch text
%
% Built on: ifdraft
%
% This is to block out big blocks of text for deletion, but to keep it around
% to mine for clips.
%

\newlength{\intomarginwidth}
\addtolength{\intomarginwidth}{\linewidth}
\addtolength{\intomarginwidth}{\marginparsep}
\addtolength{\intomarginwidth}{\marginparwidth}

\newcommandx{\scratch}[2][1={Scratch},usedefault]{%
    \ifoptionfinal{%
        % nothing
    }{{%
        \leavevmode\color{black!030}%
        \rule{\intomarginwidth}{0.4pt}
        \if\relax\detokenize{#1}\relax%
            %nothing
        \else%
            \marginpar{\small{\textsc{#1}}}
        \fi%
        #2
        \rule{\intomarginwidth}{0.4pt}
    }}%
}



%======================================================================
% Draft-only content
%
% Built on: ifdraft
%
% These macros mark off content that should NOT appear
% in the final version of the document.
%
% If you use the 'final' option in the 'documentclass' macro:
%
%       \documentclass[final]{article}
%
% then the content will completely disappear.
%

% DEPRECATED. Use scratch instead
\newcommand{\draftonlysection}[2]{
    \ifoptionfinal{%
        %nothing
    }{{%
        \section*{#1 (draft only)}

        #2

        \textbf{*** This section will not be present in the final version of the document ***}
    }}%
}



%======================================================================
% Document version and history
%
% This creates a draft-only section (see above) that shows Git history of the
% document.
%
% This works by including a TeX snippet generated by the 'gen_meta_tex.sh'
% script, which is called by the Makefile during the make process.
%
% See the 'gen_meta_tex.sh' script for details of what is defined, and the
% Makefile for details of how the script is called.
%
% Note that the Git log is a separate plaintext document, 'doc-git-log.txt',
% that is also generated during the build process (see the Makefile). This is
% because of how tricky verbatim text is in LaTeX.
%
% It's much easier to print a plain text file verbatim than it is to figure out
% how to define a multi-line variable and print it verbatim.
%

\newcommand{\documenthistory}{

    \draftonlysection{Document Version and History}
    {
        Document date: \GitAuthorDate \\
        Git source version: \texttt{\GitAbrHash}%
        \ifnum \GitNumDirty > 0%
            , \space plus uncommitted changes to \GitNumDirty \space files
        \fi

        Recent history:
        \lstinputlisting{doc-git-log.txt}

        \ifcsname GitUrl\endcsname
            Repository URL: \\
            \GitUrl
        \fi
    }

}


%======================================================================
% Deprecated
\newcommand*{\written}[1]{\outline{#1}}
\newcommand*{\draftonlynote}[1]{\scratch[]{#1}}
\newcommand*{\draftonlymargin}[1]{\draftnote{#1}}
